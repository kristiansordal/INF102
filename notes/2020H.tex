\documentclass{article}
\input{preamble.tex}

\begin{document}
\section{Eksamen Høst 2020}
    \subsection{Oppgave 1}
    Hva er kjøretiden til denne koden?

    \begin{lstlisting}
public double interestOnLoan(double amount, int n) {
    amount = amount * 1.01; 
}
return amount
    \end{lstlisting}

    \begin{ans}
        \[ O(n) \]
    \end{ans}

    \subsection{Oppgave 2}
    Hva er kjøretiden til denne koden?

    \begin{lstlisting}
public static int countOneBits(int n) {
    int bits = 0;
    while (n > 0) {
        if (n % 2 == 1) {
            bits++;
        }
        n = n / 2;
    }
    return bits;
}
    \end{lstlisting}

    \begin{ans}
        \[ O\left( \log n  \right) \]
    \end{ans}

    \subsection{Oppgave 3}
    Hva er kjøretiden til denne koden?
        
    \begin{lstlisting}
public static int countSteps (int n) {
    int pow = 2;
    int steps = 0;

    for(int i = 0; i < n; i++) { // n
        if (i == pow) { // n
            pow *= 2;
            for(int j = 0; j < n; j++) {
                steps++;
            }
        } else {
            steps++;
        }
    }
    return steps;
} 
    \end{lstlisting}

    \begin{ans}
        \[ O\left( n \log n \right) \]
    \end{ans}

    \subsection{Oppgave 4}
    Hva er kjøretiden til denne koden?
    \begin{lstlisting}
public staic String makeRandomString(int n) {
    String ans = "";
    for (int i = 0; i < n; i++) {
        char c = (char) ('a'+26*Math.random());
        ans += c;
    }
    return ans;
    
} 
    \end{lstlisting}

    \begin{ans}
        \[ O(n^2) \]
    \end{ans}

    \subsection{Oppgave 5}
    Hva er kjøretiden til denne koden?

    \begin{lstlisting}
public static double computeAreaUnderCurve(LinkedList<Double> y) {
    Double area = 0.0;

    for(int i = 1; i < y.size(); i++) {
        area = area + (y.get(i - 1) + y.get(i)) / 2;
    }
    return area;
}
    \end{lstlisting}

    \begin{ans}
        \[ O(n^2) \]
    \end{ans}

    \subsection{Oppgave 6}
    Hva er kjøretiden til denne koden?
    \begin{lstlisting}
// n = list.size() 
public static Collection<Integer> findLargestK(ArrayList<Integer> list, int k) {
    PriorityQueue<Integer> pq = new PriorityQueue<Integer>();
    for(int num: list) { // n
        if(pq.size() < k || pq.peek() < num) { // 1
            pq.add(num) // log k
        }

        if(pq.size() > k) { // 1
            pq.poll(); // log k
        }
    }
    return pq;
}
    \end{lstlisting}
    Har kan du anta at \( k < n \).

    \begin{ans}
        \[ O\left( n \log\left( k \right) \right) \]
    \end{ans}

    \subsection{Oppgave 7}
    Denne koden kalles når \texttt{words} er tom og vil da fylle inn settet \texttt{words}. Hva er kjøretiden til denne koden?

    \begin{lstlisting}
HashSet<String> dictionary;        
HashSet<String> words;

// n = dictionary.size()
// k = letters.length()

private void possibleWords(String word, String letters) {
    if(!word.isBlank() && dictionary.contains(word)) { // n
        words.add(word); // 1
    }

    int k = letters.length(); // 1

    for(int i = 0; i < k; i ++) { // k
        // add letter to word
        String nextWord = word + letters.charAt(i); // k

        // remove letter from available letters
        String nextLetters = letters.substring(0, i); 

        if (i < k - 1) {
            nextLetters = nextLetters + letters.substring(i + 1, k);
        }

        // continue searching
        possibleWords(nextWord, nextLetters);
    }
}
    \end{lstlisting}

    Her kan du anta at \( k < n \).

    \begin{ans}
        \[ O\left( k! \cdot  k  \right) \]
        
    \end{ans}

    \subsection{Oppgave 8}
    Forklar hvordan algoritmen \texttt{QuickSort} ville sortert denne listen med tall.

    \[ 4,7,15,1,9,3,6,12,2 \]

    \begin{ans}

        \texttt{QuickSort} bruker en \texttt{pivot} for å dele opp listen i verdier som er større enn og mindre enn denne verdien. Sorteringen vil skje på følgende måte: Som første pivot vil vi bruke veriden lengst til høyre i listen. Høyre og venstre er de indeksene i listen som vi foreløpig ser på, og er initalisert til 0 og lengden av listen - 1, som i dette tilfellet er 8.

        \begin{table}[H]
            \begin{center}
                \begin{tabular}[c]{|l|l|l|l|}
                    \hline
                    Liste&Pivot&Høyre&Venstre  \\
                    \hline
                     4,7,15,1,9,3,6,12,2&2&0&9  \\
                    \hline
                     1,2,4,3,6,7,9,12,15&7&2&8 \\
                    \hline
                     1,2,4,3,6,7,9,12,15&6&2&4 \\
                    \hline
                     1,2,3,4,6,7,9,12,15&3&2&3 \\
                    \hline
                     1,2,3,4,6,7,9,12,15&15&6&8 \\
                    \hline
                     1,2,3,4,6,7,9,12,15&12&6&7 \\
                    \hline
                \end{tabular}
            \end{center}
        \end{table}
        
    \end{ans}

    \subsection{Oppgave 9}
    DFS (dybde først søk) er en metode for å besøke alle nodene i en graf. Hver node blir kun besøkt en gang. Forklar hvordan DFS kjører på denne grafen ved å stegvis beskrive hva algoritmen gjør og i hvilken rekkefølge nodene vil bli besøkt. Du skal begynne søket i node "a". Det er mange rette rekkefølger, du skal bare gi en mulig rekkefølge for DFS.

    \begin{figure}[H]
        \begin{center}
            \includegraphics[width=0.45\textwidth]{dfsgraph9}
        \end{center}
    \end{figure}

    \begin{ans}
        DFS opererer på \textit{Last in, First out} prinsippet. Den vil derfor besøke alle nodene i siten til et løv, før den går videre opp i grafen og finner andre noder. Vi kan beskrive rekkefølgen og hva algoritmen gjør i en tabell.

        \begin{table}[H]
            \begin{center}
                \begin{tabular}[c]{|l|l|l|l|}
                    \hline
                    Nåværende node& Neste node & Kø & Besøkte  \\
                    \hline
                      a&b&c,b& \\
                    \hline
                      b&h&c,f,h&a \\
                    \hline
                      h&e&c,f,e&a,b \\
                    \hline
                      e&g&c,f,f,g&a,b,h \\
                    \hline
                      g&f&c,f,f&a,b,h,e \\
                    \hline
                      f&c&c,f,c&a,b,h,e,g \\
                    \hline
                      c&d&c,f,d&a,b,h,e,g,f \\
                    \hline
                      d&&c,f&a,b,h,e,g,f,c \\
                    \hline
                      &&&a,b,h,e,g,f,c,d \\
                    \hline
                \end{tabular}
            \end{center}
        \end{table}
    \end{ans}

\subsection{Oppgave 10}
BFS (bredde først søk) er en metode for å besøke alle nodene i en graf. Hver node blir kun besøkt en gang. Forklar hvordan BFS kjører på denne rettede grafen ved å beskrive hvilken rekkefølge nodene vil bli besøkt. Du skal begynne søket i node "a". Det er mange rette svar du skal bare gi en mulig rekkefølge for BFS.



\begin{figure}[H]
    \begin{center}
        \includegraphics[width=0.45\textwidth]{bfsgraph10}
    \end{center}
\end{figure}

\begin{ans}
    BFS opererer på \textit{First in, First out} prinsippet, og vi kan beskrive hva algoritmen gjør med en tabell
        \begin{table}[H]
            \begin{center}
                \begin{tabular}[c]{|l|l|l|l|}
                    \hline
                    Nåværende node& Neste node & Kø & Besøkte  \\
                    \hline
                      a&d&d&\\
                    \hline
                      d&g&g,h&a\\
                    \hline
                      g&h&h,c&a,d\\
                    \hline
                      h&c&c&a,d,g\\
                    \hline
                      c&f&f,a&a,d,g,h\\
                    \hline
                      f&e&a,e&a,d,g,h,c\\
                    \hline
                      e&&a&a,d,g,h,c,f\\
                    \hline
                      &&&a,d,g,h,c,f,e\\
                    \hline
                \end{tabular}
            \end{center}
        \end{table}
\end{ans}
\end{document}
