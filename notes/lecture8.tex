\documentclass{article}
\input{preamble.tex}

\begin{document}
    \section{Forelesning 8}

    \subsection{Set datastruktur}

    \begin{itemize}
        \item Set er nesten list, bare at i set er alle elementer unike.
        \item Viktige metoder
            \begin{itemize}
                \item Add
                \item Remove
                \item Contains
            \end{itemize}
        \item Set i Java:
            \begin{itemize}
                \item TreeSet
                \item HashSet
            \end{itemize}
    \end{itemize}

    % \subsubsection{Set kjøretid}

    % \begin{table}[H]
    %     \begin{tabular}[c]{|c|c|c|}
    %         \hline
    %          &&  \\
    %          &&  \\
    %          &&  \\
    %         \hline
    %     \end{tabular}
    % \end{table}

    \subsection{Binary search tree}

    \begin{itemize}
        \item Binært søketre på norsk
        \item Den datastrukturen som er brukt i TreeSet
        \item Ligner veldig på Heap
        \item Kan kun brukes med Comparable
        \item Data invariant:
        \begin{itemize}
        \item Verider til venstre er mindre enn roten
        \item Verdier til høyre er større enn roten
        \end{itemize}
    \end{itemize}

    \subsubsection{Balansert}
    \begin{itemize}
        \item Siden kjøretid er høyden av treet må vi prøve å holde høyden til \( O\left( \log\left( n \right) \right) \)
        \item Balansering kan gjøres på mange måter.
        \item Beskrives i kapittel 3.3 i boken
        \item Hvis man finner en node der forskjellen mellom antall noder til venstre og antall noder til høyre er stor, da må man flytte fra høyre til venstre
        \item Det å balansere et tre kan ta litt mer en \( O\left( \log\left( n \right) \right) \) tid
        \item Som i ArrayList sin "grow" metode, trengs balansering ikke gjøres hver gang.
        \item Detaljene i 3.3 er ikke relevant ofr eksamen, men viktig å forstå at balansering er nødvendig for \( \log\left( n \right) \) kjøretid.
    \end{itemize}

    \subsection{HashSet}

    \begin{itemize}
        \item Kan brukes på alle objekter i Java
        \item Krever metoden \texttt{hashCode()}
            \begin{itemize}
                \item Mer om den senere
            \end{itemize}
        \item Bruker mer minne, men er rask
        \item Ikke effektivt hvis du trenger å finne max/min
    \end{itemize}
\end{document}
