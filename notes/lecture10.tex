\documentclass{article}
\input{preamble.tex}

\begin{document}
    \section{Forelesning 10}

    \subsection{Graf}
    Grafer er en datastruktur som består av
    \begin{itemize}
        \item Noder (Vertices)
        \item Kanter (Edges) - par av noder
    \end{itemize}

    Binære trær er grafer.

    \subsection{Hva kan grafer representere?}
    \begin{table}[H]
        \begin{center}
            \begin{tabular}[c]{|l|l|}
                \hline
                 \textbf{Noder}&\textbf{Kanter}  \\
                \hline
                 Byer& Veier  \\
                 Person& Vennskap  \\
                 Datamaskin& Nettverkskabel  \\
                 Spill posisjon& Gyldig trekk  \\
                 Student, emnet& Tar emnet  \\
                 Robot, Job& Assigned  \\
                 Lag& Kamp  \\
                \hline
            \end{tabular}
        \end{center}
    \end{table}

    \subsubsection{Gradtall}
    Antall kanet som inneholder noden

    \subsubsection{Naboskap}
    Alle noder som kan nås med en kant

    \[ N\left( b \right)=\left\{ a,c,f \right\} \wedge N\left[ b \right] = \left\{ a,b,c,f \right\}\]

    \subsubsection{Sykel}
    Noder og kanter som gjør at du kan gå fra en node og tilbake til seg selv mens du bare besøker hver node i sykelen akkurat en gang.

    \subsubsection{Tre}
    Et tre er en sammenhengende graf uten sykler.


    \subsection{Graf datastruktur}
    \begin{itemize}
        \item Viktige metoder i \texttt{Graph<V,E>}
            \begin{itemize}
                \item \texttt{boolean adjacent (V a, V b)}
                \item \texttt{Iterable<V> vertices()}
                \item \texttt{Iterable<E> edges()}
                \item \texttt{List<V> neighbours(V v)}
                \item \texttt{add/remove metoder?}
            \end{itemize}
    \end{itemize}

    \subsubsection{Kjøretid i grafer}
    \begin{itemize}
        \item Kjøretid i grafer avhenger av 2 parametere
            \begin{itemize}
                \item \( N \) er antall noder
                \item \( M \) er antall kanter
            \end{itemize}
        \item Noen ganger er man mer presis
            \begin{itemize}
                \item \( D \) = degree(v)
                \item \( L \) = length of longest cycle
                \item \( C \) = number of components
            \end{itemize}
    \end{itemize}

    \subsubsection{Graf datastruktur - Kantlise}

    \begin{figure}[H]
        \begin{center}
            \includegraphics[width=0.55\textwidth]{edgelist}
        \end{center}
    \end{figure}

    \begin{table}[H]
        \begin{center}
            \begin{tabular}[c]{|l|l|}
                \hline
                \textbf{Metode} & \textbf{Kjøretid} \\
                Ajacent & \( O\left( m \right) \) \\
                Vertices & \( O\left( n \right) \) \\
                Edges & \( O\left( m \right) \) \\
                Neighbours & \( O\left( m \right) \) \\
                addVertex & \( O\left( 1 \right) \)\\
                addEdge & \( O\left( 1 \right) \) \\
                \hline
            \end{tabular}
        \end{center}
    \end{table}

    \subsubsection{Graf datastruktur - Naboliste}

    \begin{figure}[H]
        \begin{center}
            \includegraphics[width=0.55\textwidth]{neighlist}
        \end{center}
    \end{figure}

    \begin{itemize}
        \item For hver node, hold liste av naboer
        \item Hold en liste av noder \texttt{List<V>}
    \end{itemize}

    \begin{table}[H]
        \begin{center}
            \begin{tabular}[c]{|l|l|}
                \hline
                \textbf{Metode} & \textbf{Kjøretid} \\
                \hline
                 Adjacent& \( O\left( \text{degree}  \right) \)  \\
                 Vertices& \( O\left( n \right) \)  \\
                 Edges& \( O\left( m \right) \)  \\
                 Neighbours& \( O\left( \text{degree} \right) \)  \\
                 addVertex& \( O\left( 1 \right) \)  \\
                 addEdge& \( O\left( 1 \right) \)  \\
                \hline
            \end{tabular}
        \end{center}
    \end{table}

    \subsubsection{Graf datastruktur - Nabomatrise}

    \begin{figure}[H]
        \begin{center}
            \includegraphics[width=0.55\textwidth]{neighmat}
        \end{center}
    \end{figure}

    \begin{itemize}
        \item 2 dimensjonell boolean tabell
        \item Bruker mye minne
    \end{itemize}

    \begin{table}[H]
        \begin{center}
            \begin{tabular}[c]{|l|l|}
                \hline
                \textbf{Metode} & \textbf{Kjøretid} \\
                \hline
                 Adjacent& \( O\left(1\right) \)  \\
                 Vertices& \( O\left( n \right) \)  \\
                 Edges& \( O\left( n^2 \right) \)  \\
                 Neighbours& \( O\left(n \right) \)  \\
                 addVertex& \( O\left( n^2 \right) \) or \( O\left(n\right) \)  \\
                 addEdge& \( O\left( 1 \right) \)  \\
                \hline
            \end{tabular}
        \end{center}
    \end{table}

    \subsection{Sammenhengende komponenter}
    Hvordan kan vi finne alle sammenhengende komponenter? Vi skal nå se på en algoritme som finner sammenhengende komponenter i en graf.

    \begin{enumerate}
        \item Søke i grafen med noe vi kaller bredde først søk.
        \item Søke i grafen med noe vi kaller dybde først søk.
        \item Bruke Union-Find (vil lære om dette snart)
    \end{enumerate}
\end{document}
