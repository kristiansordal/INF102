\documentclass{article}
\input{preamble.tex}

\begin{document}
    \section{Forelesning 6}

    \subsection{Divide \& Conquer}
    \begin{itemize}
        \item En nyttig strategi når man skal finne ffektive algoritmer.
        \item Divide
            \begin{itemize}
                \item Del opp input og løs hver del for seg
            \end{itemize}
        \item Conquer
            \begin{itemize}
                \item Finn en måte å slå sammen de to delene til løsning for hele input
            \end{itemize}
    \end{itemize}

    \subsection{Mergesort}
    \begin{itemize}
        \item Divide
            \begin{itemize}
                \item Lag to lister som hver har halvparten av elementene
                \item Sorter de to små listene
            \end{itemize}
        \item Conquer
            \begin{itemize}
                \item Gitt to små sorterte lister, slå disse sammen til en stor sortert liste.
            \end{itemize}
            \item La oss implementere mergesort.
    \end{itemize}

    \subsection{Nedre grense på kjøretid}
    \begin{itemize}
        \item Vi kan se å sortering som å velge den rette ordningen blant alle mulige ordninger av en liste.
        \item Hvor mange måter er det å ordne \( n \) elementer? (\( n! \))
        \item Tenk deg en liste av alle mulige ordninger
        \item Gjør en sammenligning av 2 elementer \( a \) og \( b \).
        \item Blant alle ordninger i listen kan vi
        \begin{itemize}
            \item Enten Krysse ut de der \( a < b \)
            \item Eller krysse ut de der \( b < a \)
        \end{itemize}

        \item Hvor mange ganger må vi dele på 2 før \( n! \) blir 1?
            \item \( \log\left( n! \right) \)
    \end{itemize}

    \subsection{Sorter \( n \) heltall}
    \begin{itemize}
        \item Hvor for kan vi sortere \( n \) heltall der alle tallene er
            \begin{itemize}
                \item Mellom 0 og 1000
                \item Mellom 0 og \( n \)
                \item Mellom 0 og \( n^2 \)
            \end{itemize}
    \end{itemize}

    \subsection{Bucket Sort}
    \begin{itemize}
        \item Ikke egentlig en sorteringsalgoritme
            \item Deler opp i grupper (buckets/bins)
                \begin{itemize}
                    \item For alle par \( a,b \) har vi: \( a \leq b \leftrightarrow \text{bucket}(a) \leq \text{bucket}(b) \)
                \end{itemize}
                \item Trenger funksjon for bucket nummer
                    \begin{itemize}
                        \item Eksempel: String, 26 buckets, \( a \) i første, \( b \) i andre.
                    \end{itemize}
                    \item Radix sort ligner på bucket sort
                        \begin{itemize}
                            \item Deler igjen opp hver bucket basert på 2. bokstav.
                        \end{itemize}
    \end{itemize}
\end{document}
