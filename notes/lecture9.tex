\documentclass{article}
\input{preamble.tex}

\begin{document}
    \section{Forelesning 9}
    \subsection{Læringsmål}

    \begin{itemize}
        \item Hva er programmering
            \begin{itemize}
                \item Å sette sammen enkle operasjoner til å utføre en mer komplisert oppgave
            \end{itemize}
        \item Et godt program er:
            \begin{itemize}
                \item Rett: Læres i INF100
                \item Oversiktlig: Læres i INF101
                \item Effektivt: Læres i INF102
            \end{itemize}
        \item Problem \( \rightarrow \) Algoritme
            \begin{itemize}
                \item Teoretisk analyse av problem
                \item Beskrive algoritme med ord
                \item Beregne kjøretid med Big \( O \)
            \end{itemize}
        \item Algoritme \( \rightarrow \) Java kode
            \begin{itemize}
                \item Kjenne og bruke datastrukturer rett
                \item Kjenne kjøretid til funksjoner i Java
            \end{itemize}
    \end{itemize}

    \subsubsection{Kunnskaper}
    \begin{itemize}
        \item Matematikken bak analyse av algoritmer når det gjelder kjøretid og minnebruk
            \begin{itemize}
                \item \color{red}{Big \( O \) (Kapittel 1)}
            \end{itemize}
        \item Sentrale algoritmer og datastrukturer for sortering
            \begin{itemize}
                \item \color{red}{Mergesort, Quicksort og Heap (Kapittel 2)}
            \end{itemize}
        \item Sentrale algoritmer og datastrukturer for søking
            \begin{itemize}
                \item \color{red}{Binærsøk + Heap (Kapittel 3)}
            \end{itemize}
        \item Viktige graf algoritmer og deres datastrukturer
            \begin{itemize}
                \item \color{red}{Kapittel 4}
            \end{itemize}
    \end{itemize}

    \subsubsection{Ferdigheter}
    \begin{itemize}
        \item Kan matematisk analysere algoritmers kjøretid og minnebruk
            \begin{itemize}
                \item \color{red}{Big \( O \)}
            \end{itemize}
        \item Kan empirisk analysere algoritmers kjøretid og minnebruk
            \begin{itemize}
                \item \color{red}{Kjøre programmet og ta tiden, doubling rate}
            \end{itemize}
        \item Kan programmere effektive algoritmer for sortering, søking og grafer
            \begin{itemize}
                \item \color{red}{Obligene og ukesoppgavene}
            \end{itemize}
        \item Kan gjenkjenne og løse nye problem med teknikker lært i dette emnet og utforme nye algoritmer for lignende problem
            \begin{itemize}
                \item \color{red}{Obligene}
            \end{itemize}
        \item Kan bruke generisk programmering til å programmere algoritmer
            \begin{itemize}
                \item \color{red}{Se på eksempler fra forelesningene og obligenee}
            \end{itemize}
    \end{itemize}

    \subsubsection{Generell kompetanse}
    \begin{itemize}
        \item Diskutere algoritmer og datastrukturer med andre
            \begin{itemize}
                \item \color{red}{Gruppetimer, samarbeid på oblig}
            \end{itemize}
        \item Er bevisst på forskjell på effektivitet av programvare
            \begin{itemize}
                \item \color{red}{Dekket første uke}
            \end{itemize}
        \item Anvende kkunnskaper og ferdigheter om algoritmer og datastrukturer på ulike anveldelsesdomener
            \begin{itemize}
                \item \color{red}{Problemløsning, ukesoppgaver}
            \end{itemize}
    \end{itemize}

    \[ 10 \cdot  \log\left( 10 \right) = 10, \quad n = 10 \]
    \[ 10 \cdot 10 = 100, \quad n, k = 10 \]

\end{document}
