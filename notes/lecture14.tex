\documentclass{article}
\input{preamble.tex}

\begin{document}
    \section{Forelesning 14}
    \subsection{Rettede grafer}

    \begin{figure}[H]
        \begin{center}
            \includegraphics[width=0.95\textwidth]{directedgraph}
        \end{center}
        \caption{Eksempel på en rettet graf}
        \label{fig:directedgraph}
    \end{figure}

    Noder som kun har utgående kanter kalles \texttt{source}, og noder som kun har inngående noder kalles \texttt{sink}.

    \subsubsection{Nye metoder i rettede grafer}

    \begin{itemize}
        \item \texttt{Iterable<V> outNeighbours (V v)}
        \item \texttt{Iterable<V> inNeighbours (V v)}
        \item \texttt{int outDegree}
    \end{itemize}

    \subsubsection{BFS og DFS i rettet graf}
    \begin{itemize}
        \item Virker akkurat som i urettet graf
        \item Der du før brukte \texttt{neighbours()}, må du nå bruke \texttt{outNeighbours()}
        \item DFS kan burkes til å finne en rettet sykel
            \begin{itemize}
                \item Ser dere hvordan?
                \item Jo, hvis vi kommer til en node som allerede ligger i \texttt{found}.
            \end{itemize}
    \end{itemize}

    \subsection{Strongly connected components}
    \begin{itemize}
        \item Forkortes SCC
        \item Det finnes sti fra \( a \) til \( b \) og sti fra \(  b \) til \( a \), hvis og bare hvis \( a \) og \( b  \) er i samme SCC.
        \item Det vil si at en strongly connected component er et sett med noder der det alltid finnes en sti du kan følge for å komme dit du vil.
    \end{itemize}

    \subsection{Directed Acyclic Graph}
    \begin{itemize}
        \item En graf uten noen rettede sykler kaller vi en DAG.
        \item Alle nodene i en DAG kan ordnes slik at alle kanter går fra venstre til høyre.
        \item Hvordan kan vi finne en slik ordning?
    \end{itemize}

    \subsubsection{Topological sort (finn ordning)}
    \begin{lstlisting}
    int i = 0
    while (g has a source)
        let v be a source 
        order(i) = v
        i++
        remove v from g
    \end{lstlisting}


    \subsubsection{Topological sort kjøretid}
    \begin{itemize}
        \item Kan vi finne neste source fortere enn O(n)?
            \begin{itemize}
                \item Ja, ha en liste over alle som har gradtall 0.
                \item Hver gang gradtall oppdateres
                    \begin{itemize}
                        \item Hvis ny verdi er 0, legg til i liste
                    \end{itemize}
                \item Total \( O(m) \)
                    \begin{itemize}
                        \item Oppdatering \( O(m) \)
                        \item \( O(n) \) flyttinger til/fra liste
                    \end{itemize}
            \end{itemize}
    \end{itemize}
\end{document}
