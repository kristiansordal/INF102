\documentclass[twocolumn]{article}
\input{preamble.tex}
\begin{document}
\section{Forelesning 1}

\subsection{Hva er en Algoritme?}

En algoritme er en plan for å løse et problem, ikke helt det samme som en implementasjon. Den tar en input og fir en output (metode i \textit{Java}).
\bigskip

\textit{Dagens Problem:} Gitt en liste med brukernavn, sjekk at ingen brukernavn er like.

\subsection{Forelesninger}
Forelesningene blir tatt opp og gjort tilgjengelig på Mitt UiB, og kode blir lagt ut på git.

\subsection{Pensum}
Algorithms by Robert Sedgewick \( 4^{\text{th}} \) edition. Kapittel 1-4 er pensum.

\begin{enumerate}
    \item Fundamentale Algoritmer og datastrukturer
    \item Sortering
    \item Søking
    \item Grafer
\end{enumerate}

\subsection{Beregne summen fra 1 til  n }
Følgende formel brukes for å regne summen fra 1 til \( n \).

\[ 1+2+3+\cdots + n = \sum_{i}^{n} \frac{n \cdot \left( n+1 \right)}{2} \]

Kjøretiden til implementasjonen kan estimeres til

\[ \frac{n \cdot \left( n + 1 \right)}{2}\cdot  20 \approx 10n^2 \]

\begin{align*}
    10 \cdot  n^2 &= 10^{9} \\
                  &\approx10000
\end{align*}

Kjøretid på et tilsvarende program som bruker \verb!Collections.sort()! vil ha en kjøretid på ca. \[ 10 \cdot  n \log n + 10 \cdot  n = 10^{9} \]

Når \( n \) blir stor, er alltid \( n \log n  \) raskere enn \( n^2 \).

\subsection{Hvordan forenkle beregninger?}
Det er tidkrevende å kjøre programmet og ta tiden. Det er vanskelig å vite nøyaktig hvor mange operasjoner det er, derfor velger vi å beholde største ledd, og vi ser bort ifra konstanter. Som eksempel kan følgende formel forenkles.

\[ \frac{n \cdot  \left( n + 1 \right)}{2} \cdot  20 = \sim 10 n^2 = O\left( n^2 \right) \]

\end{document}
